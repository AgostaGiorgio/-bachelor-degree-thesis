Lo scenario di simulazione prevede un'area utile ai nodi di 100x100 metri su cui 139 dei 140 nodi totali sono distribuiti in modo randomico uniforme. \\
Il sink node (per convenzione il nodo 0) è l'unico nodo della rete con posizione fissa, in particolare è posizionato al centro dell'area (x=50, y=50). \\
Ovviamente il 100\% dei nodi è in grado di generare pacchetti DATA, fatta eccezione per il sink che si limita a riceverli.\\

Il data rate del canale, per la radio principale,  è impostato a 250 Kbps. \\
Il modello energetico considerato è quello del nodo sensoriale \textit{MagoNode}++\cite{magonode++}, ideato da SensesLab e basato sulla precedente versione \textit{MagoNode}\cite{magonode}, una piattaforma ideata per WSN che fornisce supporto a tecnologie di wake-up radio e di energy harvesting.\\

Ogni nodo della rete ha 3 diverse modalità per l'antenna principale (e ognuna ovviamente prevede un consumo più o meno alto). In particolare vi è la modalità di invio (\textit{TX}), la modalità di ricezione (\textit{RX}) e la modalità in cui l'antenna consuma molto meno energia (\textit{SLEEP}). \\

Per tutte le simulazioni effettuate, l'antenna principale è in grado di coprire un'area di raggio 60m, mentre per l'antenna di wake-up l'area coperta ha raggio 25m.
Per quanto riguarda l'energy harvesting, vengono utilizzate delle celle solari che permettono appunto di accumulare energia. \'E possibile predire la quantità di energia accumulata da ogni nodo grazie a un sistema di energy prediction che, dopo un periodo di \textit{"training"} è in grado di fornire una stima affidabile dell'energia che sarà accumulata nel breve futuro.\\

\begin{table}[h]
    \setlength\doublerulesep{1mm} 
    \begin{tabular}{ |p{8cm}|p{4cm}|  }
        \hline
        \multicolumn{2}{|c|}{\textbf{Parametri di simulazione}}\\
        \hline \hline
        \textbf{Parametro}  & \textbf{Valore}\\
        \hline
        Tempo di simulazione                            & 72h \\
        Durata allenamento degli energy predictor       & 71h \\
        Tempo utile allo scambio di pacchetti           & 1h \\
        Nodi della rete                                 & 140 \\
        Nodi che possono generare pacchetti Data        & 139 \\
        Dimensione area di simulazione                  & 100 x 100 \textit{ mt\textsuperscript{2}}\\
        Distribuzione dei nodi (1-139)                  & randomica uniforme \\
        Posizione Sink (0)                              & coord(50, 50) \\
        Jitter massimo previsto                         & 100ms \\
        Piattaforma hardware                            & MagoNode++ \\
        Raggio di azione radio principale               & 60mt \\
        Raggio di azione radio wake-up                  & 25mt \\
        Frequenza radio principale                      & 2.4GHz \\
        Frequenza radio wake-up                         & 868MHz \\
        Frequenza aggiornamento wake-up addresses       & 300s \\
        Sorgente energetica esterna                     & Energia solare \\
        Capacità massima delle batterie                 & 245000mA \\
        \hline
    \end{tabular}
    \centering
    \vspace*{5mm}
    \caption{Tabella di configurazione delle simulazioni.}
    \label{configs}
\end{table}