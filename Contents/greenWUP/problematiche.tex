In questo paragrafo descriverò problematiche derivanti da mie osservazioni o da determinati risultati ottenuti durante il testing della versione base del protocollo.\\

Una prima osservazione ricade sulla fase di Relay-Selection e quindi sullo scambio dei pacchetti RTS/CTS con i wake-up message e tempi che ne derivano. GreenWUP, infatti, prevede che ad ogni nuovo pacchetto data, il nodo sender debba scegliere tra i propri vicini il next-hop che si occuperà di inoltrare il pacchetto verso il sink. Ovviamente questa è la procedura che più richiede tempo e spreco di energia per cui ripeterla sempre potrebbe essere inefficace soprattutto in scenari con traffico denso. Potrebbe, quindi, avere senso tenere traccia in qualche modo delle varie scelte che i nodi sender fanno durante l'intera simulazione in modo da ridurre il numero di ripetizioni di questa procedura.\\

Una seconda osservazione, strettamente collegata alla prima, ricade sul fatto che dall'altra parte i nodi receiver, una volta ricevuta una stringa di wake-up, non hanno nessuna logica decisionale riguardo l'attivarsi o meno in ascolto, infatti, da protocollo attivano la main radio in RX.\\
Ovviamente ciò comporta che in fase di Relay-Selection, ad esempio, tutti i receiver partecipano alla fase e, dato che solo uno poi sarà scelto come next-hop, tutti gli altri nodi non faranno altro che sprecare energia inutilmente. Se invece potessero decidere di non partecipare alla fase, evitando quindi di settare la radio principale a RX, potrebbero risparmiare energia. Ovviamente questo è un aspetto abbastanza complesso su cui bisogna prestare attenzione nello scegliere le condizioni con il quale un nodo decide o meno di attivarsi in RX o meno, in quanto potrebbe capitare che nessun nodo decida di partecipare alla fase di Relay-Selection rischiando non solo ripetizioni della fase, ma anche la perdita del pacchetto se dovesse essere raggiunto il limite di tentativi.\\

Infine, studiando l'implementazione della versione base del protocollo si può notare come ogni nodo receiver aspetti un Jitter puramente randomico prima di inviare il pacchetto CTS al sender. Potrebbe risultare utile sfruttare al meglio questo valore calcolandolo non randomicamente ma in funzione dell'energia residua, come succede in \textbf{GreenRoutes}\cite{greenRoutes}. In questo modo il nodo sender riceverebbe il pacchetto CTS dal nodo con più energia e, dato che viene scelto il primo CTS ricevuto, si eviterebbe di scegliere come next-hop un nodo al limite della classe energetica corrente.