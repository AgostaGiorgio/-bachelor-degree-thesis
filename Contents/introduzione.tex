Durante gli ultimi anni si è assistito ad un importantissimo sviluppo del settore dell'\textbf{Internet of Things} (\textit{IoT}). Questo sviluppo ha ovviamente contribuito all'aumento del numero di dispositivi wireless collegati a questo settore. Ad oggi i dispositivi IoT ammontano a circa 7 miliardi ma si prevede il raggiungimento dei 22 miliardi di dispositivi IoT attivi entro il 2025.\\
Ovviamente l'aumentare dei dispositivi IoT ha comportato l'aumento delle loro applicazioni. Questi dispositivi trovano infatti impiego in molti ambiti che vanno dalla semplice domotica (basta osservare il fenomeno di Amazon \textit{Alexa} e di tanti altri dispositivi per le \textit{Smart Home}), al campo medico in cui aiutano il personale medico a monitorare lo stato di salute dei paziente, sino a campi strettamente scientifici in cui i dispositivi IoT sono utilizzati per osservare agenti atmosferici o per monitorare scenari.\\

Ovviamente l'impiego di questi dispositivi ha portato moltissimi vantaggi in tutti i campi in cui sono stati applicati, semplificando il lavoro svolto dal personale umano e fornendo un'efficiente automazione di attività più o meno complesse.\\

Un settore dell'IoT particolarmente di interesse, che ha posto le basi per lo sviluppo degli standard alla base di questo settore, è quello delle \textbf{\textit{Wireless Sensor Network}} (\textit{WSN}s). Le WSN sono vaste reti costituite da nodi sensori che hanno il compito di prelevare una qualsiasi informazione tramite dei sensori (temperature, vibrazioni, umidità etc..) e inoltrarla verso un destinatario.\\
Queste reti presentano vincoli e problematiche a seconda della loro applicazione.\\
Un problema comune è rappresentato dall'alimentazione dei vari nodi. Questi infatti montano delle batterie che alimentano l'hardware, per cui è molto importante prevedere una gestione ottimale dell'energia.\\

Nella stragrande maggioranza delle applicazioni, inoltre, è richiesto che la rete abbia una bassa \textbf{\textit{latenza end-to-end}} e una \textbf{\textit{lifetime}}\footnote{Tempo che intercorre dall'inizio della prima operazione fino a quando un nodo esaurisce la propria energia}. che sia il più lunga possibile.\\
Le prestazioni energetiche sono influenzate dall'\textbf{\textit{idle listening}}, ovvero, il tempo in cui i vari nodi della rete sono attivi con l'antenna principale pronta a ricevere l'informazione senza che però sia attiva una comunicazione. \\

Risulta chiaro da subito come questo fenomeno comporti uno spreco di energia notevole, per cui è molto importante ridurre il tempo passato in idle-listening il più possibile in modo da prolungare la durata delle batterie che alimentano i vari dispositivi.\\
\\
Negli ultimi decenni si sono susseguiti diversi approcci a queste problematiche.\\

Inizialmente si è adottata la tecnica del \textbf{\textit{duty cycling}} grazie al quale si riusciva a prolungare il lifetime della rete e a ridurre i tempi di idle listening. Essenzialmente si definivano delle finestre di tempo in cui il nodo poteva mettersi in ascolto di informazioni, evitando di passare molto tempo in ascolto inutilmente. Ovviamente questa soluzione migliorava sì il consumo energetico ma aumentava moltissimo anche i ritardi ent-to-end dato che i pacchetti erano ricevuti correttamente solo se trasmessi dentro queste finestre di tempo.\\

Successivamente è stata proposta una nuova tecnologia che è quella delle \textbf{\textit{wake-up Radio}}.\\ L'obbiettivo di questa nuova proposta era quello di eliminare del tutto i tempi di idle listening.\\
Ciò che prevede questa tecnologia è, infatti, che ogni nodo della rete si trovi costantemente nello stato di SLEEP\footnote{Stato in cui il nodo non sta ricevendo o trasmettendo dati} e che si attivi solo è necessario eseguire un task, come ad esempio trasmettere o ricevere dati.\\
In altre parole una vera e propria comunicazione \textit{on-demand} tra i nodi.\\
Per fare ciò un nodo può inviare l'informazione a tutti i propri vicini (che sono in SLEEP) semplicemente trasmettendo prima una sequanza di wake-up che solleciti questi ultimi ad attivarsi.\\ Ovviamente l'uso di wake-up comporta l'utilizzo di una seconda radio a bassissimo consumo energetico. Questa radio deve sempre essere attiva; infatti, l'unico compito che ha è quello di ricevere e processare le sequenze di wake-up che i nodi ricevono.\\
Dovendo essere a bassissimo consumo energetico questa seconda radio avrà un raggio d'azione molto più piccolo rispetto alla radio principale.\\

Successivamente, la soluzione delle wake-up radio è stata migliorata grazie all'uso del concetto del \textbf{\textit{Semantic addressing}}. Questo nuovo concetto permetteva non solo di assegnare a ogni nodo della rete più sequenze di wake-up, ma prevedeva che queste sequenze avessero una semantica precisa e che non fossero assegnate casualmente ai vari nodi. Grazie a questa innovazione è stato possibile dividere il gruppo di vicini di un nodo in vari sottogruppi facilmente sollecitabili tramite determinate sequenze di wake-up. In questo modo un nodo poteva inviare un dato "svegliando" solo una parte dei propri vicini, ottimizzando così ulteriormente i consumi energetici.\\

Più recentemente, invece, si è deciso di equipaggiare i nodi delle WSN con dispositivi di \textbf{\textit{Energy Harvesting}} (\textit{EH}), dando vita a quelle che sono definite come \textbf{\textit{Green WSN}}. Questi nuovi moduli \textit{EH} sono in grado di immagazzinare e fornire ulteriore energia ai nodi mediante l'uso di turbine eoliche e/o pannelli solari.\\
Come descritto in \cite{energyHarvesting}, grazie all'utilizzo di questi dispositivi sono stati risolti molti problemi collegati al consumo energetico, in particolare è stato possibile recuperare quei nodi diventati inattivi a causa dell'esaurimento energetico.\\

La combinazione di Energy Harvesting e di nodi dotati di Wake Up Radio con Semantic Addressing può consentire di operare le reti di sensori solo con energia rinnovabile ottenibile dall'ambiente circostante (questo è possibile solo per sistemi a bassissimo consumo energetico come quelli dotati di wake up radio) di fatto rendendo questi sistemi a durata operativa limitata esclusivamente dal tempo di vita delle componenti o "energy neutral".\\

Per raggiungere questo ambizioso obiettivo i miglioramenti tecnologici delle piattaforme IoT devono essere coniugati con nuovi algoritmi e protocolli che sfruttino al meglio le capacità di wake-up e/o energy harvesting.\\
In letteratura sono già stati presentati molti protocolli che fanno uso delle tecniche sopra descritte.\\
Tra questi, molti usano semantic addressing ed energy harvesting (come CTP-WUR\cite{ctp-wur}, GREENWUP\cite{greenWup}, GREENROUTES\cite{greenRoutes}) mentre altri propongono anche logiche più sofisticate come nel caso di WHARP\cite{wharp} e G-WHARP\cite{estensioneWHARP} dove si usano tecniche di ottimizzazione basate su \textit{reinforcement learning}.\newpage

Questa tesi si è rivolta allo studio di queste soluzioni. In particolare, in base all'analisi della letteratura, è stato selezionato il protocollo GreenWUP come base di partenza di uno studio prestazionale effettuato mediante simulazioni di rete estensive. Il protocollo è stato implementato nel simulatore di rete GreenCastalia. Le sue prestazioni sono state analizzate. Nel corso dell'analisi sono stati individuati miglioramenti prestazionali possibili apportando varianti nella logica del protocollo. Tali miglioramenti  sono stati quindi implementati e valutati mediante simulazione, quantificandone l'impatto sulle metriche prestazionali.\\

Il lavoro di tesi è descritto nella tesi in cinque capitoli. Il primo capitolo introduce le problematiche affrontate e sintetizza i contributi. Il secondo descrive nel dettaglio il protocollo GreenWUP. Il terzo capitolo descrive le varianti proposte. Il quarto capitolo presenta l'implementazione del protocollo e delle sue varianti nel simulatore Green Castalia e presenta gli scenari simulativi mostrando anche i risultati ottenuti. Il quinto capitolo presenta la sintesi delle attività, le conclusioni tratte ed i possibili lavori futuri.